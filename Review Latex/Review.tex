\documentclass[12pt]{article}

% Paquetes adicionales
\usepackage[utf8]{inputenc}
\usepackage[spanish]{babel}
\usepackage{natbib}
\usepackage{titling} % Para ajustar el espacio del título
\usepackage{geometry} % Para ajustar los márgenes

\setlength{\droptitle}{-3cm} % Ajusta el espacio del título según sea necesario
\geometry{margin=2.5cm} % Ajusta los márgenes según sea necesario

\title{Innovaciones en la Protección de Cultivos Cítricos mediante la Utilización de IA para la Predicción, Detección Temprana y Manejo Integrado de Plagas y Enfermedades.}
\author{Gianluca Lovatto}
\date{Fecha de presentació}

\begin{document}

\maketitle

\begin{abstract}
Resumen de todo el contenido de la Review.

\end{abstract}

\section*{Palabras clave}
Palabra clave 1; Palabra clave 2; Palabra clave 3; etc.

\section{Introducción}


\section{Inteligencia Artificial aplicada a la predicción de Plagas y Enfermedades en Cultivos Cítricos}
A que nos referimos cuando hablamos de predicción y la importancia de la predicción en el combate de plagas y enfermedades citrícolas. 



\subsection{Modelos de aprendizaje automático y análisis de datos utilizados para la predicción}
 \citep{autor_año}.

\subsection{Casos de estudio y resultados relevantes en predicción}


\section{Inteligencia Artificial aplicada a la Detección Temprana de Plagas y Enfermedades en Cultivos Cítricos}
Que es la detección temprana y breve descripción de métodos de IA utilizados.

\subsection{Sensores y tecnologías para la monitorización de cultivos con IA}
 \citep{autor_año}.

\subsection{Ejemplos de aplicaciones exitosas en detección temprana}

\section{Ayuda de la IA en el Manejo Integrado de Plagas y Enfermedades en Cultivos Cítricos}
Como se relaciona el manejo integrado con el control preventivo y predictivo. 

\section{Desafíos y Futuras Direcciones}
Consideración de obstáculos en la implementación de IA en la protección de cultivos cítricos 
Posibles mejoras y avances en tecnologías de IA para abordar estos desafíos
Perspectivas futuras y áreas de investigación prometedoras



\section{Conclusiones}
Resume los hallazgos clave de tu revisión bibliográfica y destaca las lagunas en la investigación existente. Ofrece sugerencias para investigaciones futuras.

\section*{Agradecimientos}
Si deseas, puedes incluir una sección de agradecimientos para reconocer a personas o instituciones que hayan contribuido a tu trabajo.

\section*{Referencias}
\begin{thebibliography}{99}
\bibitem[Autor et al., Año]{autor_año} Información bibliográfica de la referencia.
\end{thebibliography}

\end{document}